%==============================================================================
% Sjabloon onderzoeksvoorstel bachelorproef
%==============================================================================
% Gebaseerd op LaTeX-sjabloon ‘Stylish Article’ (zie voorstel.cls)
% Auteur: Jens Buysse, Bert Van Vreckem
%
% Compileren in TeXstudio:
%
% - Zorg dat Biber de bibliografie compileert (en niet Biblatex)
%   Options > Configure > Build > Default Bibliography Tool: "txs:///biber"
% - F5 om te compileren en het resultaat te bekijken.
% - Als de bibliografie niet zichtbaar is, probeer dan F5 - F8 - F5
%   Met F8 compileer je de bibliografie apart.
%
% Als je JabRef gebruikt voor het bijhouden van de bibliografie, zorg dan
% dat je in ``biblatex''-modus opslaat: File > Switch to BibLaTeX mode.

\documentclass{voorstel}

\usepackage{lipsum}

%------------------------------------------------------------------------------
% Metadata over het voorstel
%------------------------------------------------------------------------------

%---------- Titel & auteur ----------------------------------------------------

% TODO: geef werktitel van je eigen voorstel op
\PaperTitle{Titel voorstel}
\PaperType{Onderzoeksvoorstel Bachelorproef 2020-2021} % Type document

% TODO: vul je eigen naam in als auteur, geef ook je emailadres mee!
\Authors{Joery Haelewyck\textsuperscript{1}} % Authors
\CoPromotor{Piet Pieters\textsuperscript{2} (Bedrijfsnaam)}
\affiliation{\textbf{Contact:}
  \textsuperscript{1} \href{mailto:joery.haelewyck.w4357@student.hogent.be}{joery.haelewyck.w4357@student.hogent.be};
  \textsuperscript{2} \href{mailto:piet.pieters@acme.be}{piet.pieters@acme.be};
}

%---------- Abstract ----------------------------------------------------------

\Abstract{Hier schrijf je de samenvatting van je voorstel, als een doorlopende tekst van één paragraaf. Wat hier zeker in moet vermeld worden: \textbf{Context} (Waarom is dit werk belangrijk?); \textbf{Nood} (Waarom moet dit onderzocht worden?); \textbf{Taak} (Wat ga je (ongeveer) doen?); \textbf{Object} (Wat staat in dit document geschreven?); \textbf{Resultaat} (Wat verwacht je van je onderzoek?); \textbf{Conclusie} (Wat verwacht je van van de conclusies?); \textbf{Perspectief} (Wat zegt de toekomst voor dit werk?).

Bij de sleutelwoorden geef je het onderzoeksdomein, samen met andere sleutelwoorden die je werk beschrijven.

Vergeet ook niet je co-promotor op te geven.
}

%---------- Onderzoeksdomein en sleutelwoorden --------------------------------
% TODO: Sleutelwoorden:
%
% Het eerste sleutelwoord beschrijft het onderzoeksdomein. Je kan kiezen uit
% deze lijst:
%
% - Mobiele applicatieontwikkeling
% - Webapplicatieontwikkeling
% - Applicatieontwikkeling (andere)
% - Systeembeheer
% - Netwerkbeheer
% - Mainframe
% - E-business
% - Databanken en big data
% - Machineleertechnieken en kunstmatige intelligentie
% - Andere (specifieer)
%
% De andere sleutelwoorden zijn vrij te kiezen

\Keywords{Onderzoeksdomein. Keyword1 --- Keyword2 --- Keyword3} % Keywords
\newcommand{\keywordname}{Sleutelwoorden} % Defines the keywords heading name

%---------- Titel, inhoud -----------------------------------------------------

\begin{document}

\flushbottom % Makes all text pages the same height
\maketitle % Print the title and abstract box
\tableofcontents % Print the contents section
\thispagestyle{empty} % Removes page numbering from the first page

%------------------------------------------------------------------------------
% Hoofdtekst
%------------------------------------------------------------------------------

% De hoofdtekst van het voorstel zit in een apart bestand, zodat het makkelijk
% kan opgenomen worden in de bijlagen van de bachelorproef zelf.
%---------- Inleiding ---------------------------------------------------------

\section{Introductie} % The \section*{} command stops section numbering
\label{sec:introductie}

%Hier introduceer je werk. Je hoeft hier nog niet te technisch te gaan.

%Je beschrijft zeker:

%\begin{itemize}
  %\item de probleemstelling en context
  %\item de motivatie en relevantie voor het onderzoek
  %\item de doelstelling en onderzoeksvraag/-vragen
%\end{itemize}

De term 'Microservices' werd voor het eerst gebruikt in een evenement voor software architecten in 2011.
Deze benaming beschrijft een stijl van architectuur, die in die tijd nog in de kinderschoenen stond. Amazon en Netflix waren de pioniers van deze stijl. Deze vorm van architectuur wordt elk jaar populairder dankzij haar flexibiliteit.\\
monolitische systemen zijn grote en vaak complexe gehelen die onderling met elkaar verbonden zijn. Deze structuur vertegenwoordigd nog altijd het merendeel van de bedrijfssystemen. Hiervoor wordt meestal niet bewust gekozen maar is het resultaat van beslissingen op bedrijfsniveau.\\
Oudere bedrijven ondervinden steeds vaker de limieten van de monolitische systemen waardoor ze willen overschakelen naar de flexibelere microservices. Maar is dit wel altijd de juiste keuze? Is er wel genoeg kennis om deze overstap te maken? Hoe lang zal de transformatie duren? Is er genoeg budget? Kortom er zijn heel wat vragen die beantwoord moeten worden.
  

 

%---------- Stand van zaken ---------------------------------------------------

\section{State-of-the-art}
\label{sec:state-of-the-art}

\subsection{Monolitische systemen}
Een monolitische systeem is een architecturale stijl of een softwareontwikkelingspatroon. Vaak zijn het grote en complexe systemen die alle functionaliteit bevatten om aan de noden van de business te voldoen. De logheid is historisch gegroeid: door wijzigingen op wijzigingen. Hierdoor is het moeilijker te onderhouden en vatbaar voor fouten.
~\autocite{Monolith2014}

\subsection{Microservices}
Een microservice is een enkele applicatie die bestaat uit een reeks kleine services, die elk hun eigen proces draaien. Deze processen communiceren met 'light-weight' mechanismen, hoofdzakelijk een API. De services zijn onafhankelijk bruikbaar door volledig geautomatiseerde implementatiemachines. 
~\autocite{Microservices2014}

%---------- Methodologie ------------------------------------------------------
\section{Methodologie}
\label{sec:methodologie}

Als eerste stap zal er een grondige literatuurstudie uitgevoerd worden. In deze studie zullen verschillende aspecten van microservices en monolitische systemen behandeld worden. Het onderzoek zal op basis van volgende sleutelwoorden gestart worden: holacracy, continuous delivery, domain-Driven design, serverless, API, REST, sockets, TCP, gateway, circuit breakers, load balancer en proxy. ~\autocite{Glen2018} ~\autocite{Alshuqayran2016}
Indien er te weinig informatie uit de literatuurstudie gepuurd wordt, zal er een enquête afgenomen worden bij bedrijven die een dergelijke transformatie achter de rug hebben of er volop mee bezig zijn.\\
Aan de hand van de opgedane kennis wordt er een grondig vergelijking uitgevoerd tussen de twee architecturen. Vervolgens wordt er een model opgesteld die gebruikt kan worden om een inschatting te maken over alle mogelijke gevolgen van een transformatie van een monolitische structuur naar een systeem bestaande uit microservices. Het model zal informatie bevatten over volgende criteria:
\begin{itemize}
    \item duur
    \item financiële aspect
    \item vereiste kennis
    \item haalbaarheid
    \item schaalbaarheid
    \item beschikbaarheid
    \item onderhoudbaarheid
    \item ROI
\end{itemize} 
Tenslotte zal het opgestelde model getest worden in verschillend scenario’s waarbij een conclusie genoteerd wordt.

%---------- Verwachte resultaten ----------------------------------------------
\section{Verwachte resultaten}
\label{sec:verwachte_resultaten}

De verwachte resultaten hangen hoofdzakelijk af van de grootte van het bestaande systeem. Hoe groter de monoliet, hoe meer microservices er nodig zullen zijn. Dit geldt voor de meeste criteria die onderzocht zullen worden.\\
De duur van de transformatie hangt af van de grootte en de complexiteit. Diep in elkaar geneste programma's zullen meer tijd nodig hebben.\\
De kostprijs zal beïnvloed worden door verschillende factoren. Als er intern niet genoeg kennis aanwezig is, dan zal het bedrijf externe krachten moeten aanwerven. Een andere mogelijkheid is om te investeren in het huidige personeel. Ondertussen moet ook het bestaande systeem onderhouden worden.

%---------- Verwachte conclusies ----------------------------------------------
\section{Verwachte conclusies}
\label{sec:verwachte_conclusies}
Dit onderzoek zou een verduidelijking moeten geven op de impact van de transformatie van een monolitische systeem naar microservices. Op papier hebben microservices de meeste voordelen ten opzichte van de monolieten, maar deze voordelen zullen niet in iedere situatie opwegen tegen de kosten van de transformatie. Kleinere systemen zullen minder voordeel halen uit het feit dat elke functionaliteit van hun systeem geïsoleerd is. 
Hopelijk kan er een model opgesteld worden die kan voorspellen voor welke situatie de beste architectuur is.



%------------------------------------------------------------------------------
% Referentielijst
%------------------------------------------------------------------------------
% TODO: de gerefereerde werken moeten in BibTeX-bestand ``voorstel.bib''
% voorkomen. Gebruik JabRef om je bibliografie bij te houden en vergeet niet
% om compatibiliteit met Biber/BibLaTeX aan te zetten (File > Switch to
% BibLaTeX mode)

\phantomsection
\printbibliography[heading=bibintoc]

\end{document}
